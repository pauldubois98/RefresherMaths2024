\documentclass[]{article}

\usepackage{amsmath}
\usepackage{amsfonts}
\usepackage{amsthm}
\usepackage{amssymb}
\usepackage{mathrsfs}
\usepackage{stmaryrd}

\newcommand{\Q}{\mathbb{Q}}
\newcommand{\N}{\mathbb{N}}
\newcommand{\Z}{\mathbb{Z}}
\newcommand{\R}{\mathbb{R}}
\newcommand{\Primes}{\mathbb{P}}
\newcommand{\st}{\text{ s.t. }}
\newcommand{\txtand}{\text{ and }}
\newcommand{\txtor}{\text{ or }}
\newcommand{\lxor}{\veebar}

%opening
\title{Notes on Calculus}
\author{DSBA Mathematics Refresher 2024}
\date{}

\begin{document}
	
	\maketitle
	
	\begin{abstract}
		
	\end{abstract}
	
	
	\section{Derivatives: A Quick Reminder}
	The derivative of a function measures how the function's output value changes as the input value changes.
	For a function $f(x)$, the derivative, denoted by $f'(x)$ or $\frac{d}{dx}f(x)$, is defined as:
	$$
	f'(x) = \lim_{\Delta x \to 0} \frac{f(x+\Delta x) - f(x)}{\Delta x}
	$$
	Some common derivatives include:
	\begin{align*}
		\text{Power Rule:} & \quad \frac{d}{dx} \left( x^n \right) = nx^{n-1}, \\
		\text{Exponential Function:} & \quad \frac{d}{dx} \left( e^x \right) = e^x, \\
		\text{Trigonometric Functions:} & \quad \frac{d}{dx} \left( \sin x \right) = \cos x, \quad \frac{d}{dx} \left( \cos x \right) = -\sin x.
	\end{align*}
	
	\paragraph{Chain Rule}
	The chain rule is used to differentiate composite functions.
	 If you have a function $y = f(g(x))$, where one function is inside another, the chain rule states:
	$$
	\frac{dy}{dx} = f'(g(x)) \cdot g'(x)
	$$
	
	\noindent \textbf{Example:}
	If $y = \sin(3x)$, then:
	$$
	\frac{dy}{dx} = \cos(3x) \cdot 3 = 3\cos(3x)
	$$
	
	\paragraph{Product Rule}
	The product rule is used to differentiate the product of two functions.
	If you have a function $y = u(x) \cdot v(x)$, where $u(x)$ and $v(x)$ are both functions of $x$, the product rule states:
	$$
	\frac{dy}{dx} = u'(x) \cdot v(x) + u(x) \cdot v'(x)
	$$
	
	\noindent \textbf{Example:}
	If $y = x^2 \cdot \sin(x)$, then:
	$$
	\frac{dy}{dx} = 2x \cdot \sin(x) + x^2 \cdot \cos(x)
	$$
	
	\paragraph{Quotient Rule}
	The quotient rule is used to differentiate the quotient of two functions.
	If you have a function $y = \frac{u(x)}{v(x)}$, where $u(x)$ and $v(x)$ are both functions of $x$, the quotient rule states:
	$$
	\frac{dy}{dx} = \frac{u'(x) \cdot v(x) - u(x) \cdot v'(x)}{[v(x)]^2}
	$$
	
	\noindent \textbf{Example:}
	If $y = \frac{x^2}{\sin(x)}$, then:
	$$
	\frac{dy}{dx} = \frac{2x \cdot \sin(x) - x^2 \cdot \cos(x)}{[\sin(x)]^2}
	$$
	
	\section{Integration: A Quick Reminder}
	Integration is the "reverse process" of differentiation.
	The integral of a function represents the area under the curve of the function.
	The indefinite integral of a function $f(x)$ is denoted by:
	$$
	\int f(x) dx
	$$
	Some basic integration rules include:
	\begin{align*}
		\text{Power Rule:} & \quad \int x^n dx = \frac{x^{n+1}}{n+1} + C, \quad (n \neq -1) \\
		\text{Exponential Function:} & \quad \int e^x dx = e^x + C, \\
		\text{Trigonometric Functions:} & \quad \int \sin x dx = -\cos x + C, \quad \int \cos x dx = \sin x + C.
	\end{align*}
	where $C$ is the constant of integration
	\footnote{It is very common to forget the "$+C$".}
	\footnote{When I tutor young students, I give them push-ups when they forget it... they don't usually forget more than twice.}.
		
	\section{Integration by Parts}
	Integration by parts is based on the product rule for differentiation and is given by:
	$$
	\int u dv = uv - \int v du
	$$
	where $u = f(x)$ and $dv = g(x) dx$. Steps:
	\begin{enumerate}
		\item Choose $u$ and $dv$.
		\item Differentiate $u$ to get $du$.
		\item Integrate $dv$ to get $v$.
		\item Substitute into the formula.
	\end{enumerate}
	
	\noindent \textbf{Example:}
	$$
	\int x e^x dx
	$$
	Let $u = x$, $dv = e^x dx$, then $du = 1 \cdot dx$ and $v = e^x$. Therefore:
	$$
	\int x e^x dx = x e^x - \int e^x dx = x e^x - e^x + C = e^x(x-1) + C
	$$
	
	\section{Integration by Substitution}
	Integration by substitution is used when the integral can be transformed into a simpler form by substituting $u = g(x)$. The substitution formula is:
	$$
	\int f(g(x)) \frac{d g(x)}{dx} dx = \int f(u)\,du
	$$
	Steps:
	\begin{enumerate}
		\item Choose $u = g(x)$.
		\item Compute $du = g'(x) dx$.
		\item Rewrite the integral in terms of $u$.
		\item Integrate with respect to $u$.
		\item Substitute back $u = g(x)$ if needed.
	\end{enumerate}
	\textbf{Example:}
	$$
	\int \cos(3x) dx
	$$
	Let $u = 3x$, then $du = 3 dx$ or $dx = \frac{du}{3}$.
	The integral becomes:
	$$
	\int \cos(3x) dx = \frac{1}{3} \int \cos(u)\,du = \frac{1}{3}\sin(u) + C = \frac{1}{3}\sin(3x) + C
	$$
	
	
\end{document}
