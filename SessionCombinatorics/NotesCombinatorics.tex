\documentclass[]{article}

\usepackage{amsmath}
\usepackage{amsfonts}
\usepackage{amsthm}
\usepackage{amssymb}
\usepackage{mathrsfs}
\usepackage{stmaryrd}

\newcommand{\Q}{\mathbb{Q}}
\newcommand{\N}{\mathbb{N}}
\newcommand{\Z}{\mathbb{Z}}
\newcommand{\R}{\mathbb{R}}
\newcommand{\Primes}{\mathbb{P}}
\newcommand{\st}{\text{ s.t. }}
\newcommand{\txtand}{\text{ and }}
\newcommand{\txtor}{\text{ or }}
\newcommand{\lxor}{\veebar}

%opening
\title{Notes on Combinatorics}
\author{DSBA Mathematics Refresher 2024}
\date{}

\begin{document}
	
	\maketitle
	
	\begin{abstract}
		
	\end{abstract}
	
	
	Combinatorics is the branch of mathematics dealing with counting, arrangement, and combination of objects.
	It provides the foundation for various concepts in probability, statistics, computer science, and more.
	
	\paragraph{Choosing $n$ Times: Power $n$}
	\noindent\\
	When we choose an item from a set $n$ times \textit{\textbf{with replacement}}, each choice is independent of the previous ones.
	If the set has $k$ distinct elements, the total number of ways to make these choices is given by:
	$ k^n $
	
	\textbf{Example:}
	\noindent\\
	Suppose you have a set of 3 elements $\{a, b, c\}$.
	The number of different sequences of length 4 that can be formed by choosing from this set with replacement is:
	$ 3^4 = 81 $
	
	\paragraph{Ordering $n$ Different Items: Factorial}
	\noindent\\
	When selecting $n$ different items from a set, the number of ways to arrange these $n$ items in order is given by $n!$ (read as "n factorial"), where:
	$$
	n! = n \times (n-1) \times \cdots \times 2 \times 1
	$$
	
	\textbf{Example:}
	\noindent\\
	For a set of 5 distinct items $\{a, b, c, d, e\}$, the number of ways to arrange all 5 items is:
	$$
	5!
	= 5 \times 4 \times 3 \times 2 \times 1
	= 120
	$$
	
	\paragraph{Choosing $k$ from $n$ \textit{(in Order)}}
	\noindent\\
	When choosing $k$ elements from a set of $n$ elements (without replacement) where the order of selection matters (permutations), the number of possible arrangements is given by:
	$$
	P(n, k) = \frac{n!}{(n-k)!}
	$$
	
	\textbf{Example:}
	\noindent\\
	Consider a set of 6 elements $\{a, b, c, d, e, f\}$, and you want to select and arrange 3 elements. The number of different arrangements is:
	$$
	P(6, 3)
	= \frac{6!}{(6-3)!}
	= \frac{6 \times 5 \times 4 \times 3 \times 2 \times 1}{3 \times 2 \times 1}
	= 120
	$$
	
	\paragraph{Choosing $k$ from $n$ \textit{(Without Order)}}
	\noindent\\
	When choosing $k$ elements from a set of $n$ elements where the order of selection does not matter (combinations), the number of possible combinations is given by the binomial coefficient:
	$$
	\binom{n}{k} = \frac{n!}{k!(n-k)!}
	$$
	
	\textbf{Example:}
	\noindent\\
	Given a set of 7 elements $\{a, b, c, d, e, f, g\}$, the number of ways to choose 3 elements without regard to order is:
	$$
	\binom{7}{3}
	= \frac{7!}{3!(7-3)!}
	= \frac{7 \times 6 \times 5 \times 4 \times 3 \times 2 \times 1}{(3 \times 2 \times 1)(4 \times 3 \times 2 \times 1)}
	= 35
	$$
	
	% choose n times => power n
	% choose n different => factorial
	% choose k from n in order
	% choose k from n without order
	
\end{document}
