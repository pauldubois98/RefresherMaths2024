\documentclass[]{article}

\usepackage{amsmath}
\usepackage{amsfonts}
\usepackage{amsthm}
\usepackage{amssymb}
\usepackage{mathrsfs}
\usepackage{stmaryrd}

\newcommand{\Q}{\mathbb{Q}}
\newcommand{\N}{\mathbb{N}}
\newcommand{\Z}{\mathbb{Z}}
\newcommand{\R}{\mathbb{R}}
\newcommand{\Primes}{\mathbb{P}}
\newcommand{\st}{\text{ s.t. }}
\newcommand{\txtand}{\text{ and }}
\newcommand{\txtor}{\text{ or }}
\newcommand{\lxor}{\veebar}

%opening
\title{Notes on Differential Equations}
\author{DSBA Mathematics Refresher 2024}
\date{}

\begin{document}
	
	\maketitle
	
	\begin{abstract}
		
	\end{abstract}
	
	
	\section{First-Order Differential Equations}
	
	A first-order differential equation can often be written in the form:
	$$
	\frac{dy}{dx} + P(x) \cdot y = Q(x)
	$$
	To solve this, we use an integrating factor "$\text{IF}$" which is defined as:
	$$
	\text{IF} = e^{\int P(x) \, dx}
	$$
	Multiplying both sides of the differential equation by $\text{IF}$ gives:
	$$
	\text{IF} \cdot \frac{dy}{dx} + \text{IF} \cdot P(x) \cdot y = \text{IF} \cdot Q(x)
	$$
	The left side of the equation is now the derivative of the product $\text{IF} \cdot y$:
	$$
	\frac{d}{dx}\left[\text{IF} \cdot y\right] = \text{IF} \cdot Q(x)
	$$
	Integrating both sides with respect to $x$:
	$$
	\text{IF} \cdot y = \int \text{IF} \cdot Q(x) \, dx + C
	$$
	Finally, solve for $y(x)$:
	$$
	y(x) = \frac{1}{\text{IF}}\left[\int \text{IF} \cdot Q(x) \, dx + C\right]
	$$
	
	\paragraph{Initial Conditions}
	
	Consider the same first-order equation with an initial condition $y(x_0) = y_0$.
	After finding the general solution as in the previous section:
	$$
	y(x) = \frac{1}{\text{IF}}\left[\int \text{IF} \cdot Q(x) \, dx + C\right]
	$$
	Apply the initial condition to find $C$:
	$$
	y(x_0) = \frac{1}{\text{IF}(x_0)}\left[\int \text{IF}Q(x) \, dx \mid_{x_0} + C\right] = y_0
	$$
	Solve for $C$ and conclude.
	
	\section{(Homogeneous) Second-Order Linear Differential Equations}
	
	A second-order homogeneous linear differential equation is of the form:
	$$
	a\frac{d^2y}{dx^2} + b\frac{dy}{dx} + cy = 0
	$$
	The characteristic equation associated with this differential equation is:
	$$
	ar^2 + br + c = 0
	$$
	The nature of the solutions depends on the discriminant $\Delta = b^2 - 4ac$:
	\begin{itemize}
		\item If $\Delta > 0$, the roots $r_1$ and $r_2$ are real and distinct.
		The solution is: $y(x) = C_1e^{r_1x} + C_2e^{r_2x}$.
		\item If $\Delta = 0$, the roots are real and repeated $r_1 = r_2 = r$.
		The solution is: $y(x) = (C_1 + C_2x)e^{rx}$.
		\item If $\Delta < 0$, the roots are complex $r_{1,2} = \alpha \pm i\beta$.
		The solution is: $y(x) = e^{\alpha x}(C_1\cos(\beta x) + C_2\sin(\beta x))$.
	\end{itemize}
	
	\section{Second-Order Linear Differential Equations}
	
	For the general second-order linear differential equation:
	$$
	a\frac{d^2y}{dx^2} + b\frac{dy}{dx} + cy = g(x)
	$$
	The solution is the sum of the complementary function $y_c(x)$
	(the solution to the associated homogeneous equation)
	and a particular solution $y_p(x)$:
	$$
	y(x) = y_c(x) + y_p(x)
	$$
	
	The method for finding $y_p(x)$ depends on the form of $g(x)$:
	\begin{itemize}
		\item If $g(x)$ is a polynomial, use the method of undetermined coefficients.
		\item If $g(x)$ is of an exponential or trigonometric form, use an answer of a similar form.
		\item For other forms of $g(x)$, the method of variation of parameters can be used.
	\end{itemize}
	
	\paragraph{Initial Conditions}
	
	Given the general solution:
	$$
	y(x) = y_c(x) + y_p(x)
	$$
	And initial conditions $y(x_0) = y_0$ and $\frac{dy}{dx}\Big|_{x=x_0} = y'_0$, we determine the constants $C_1$ and $C_2$ in $y_c(x)$:
	$$
	y(x_0) = y_c(x_0) + y_p(x_0) = y_0
	$$
	$$
	\frac{dy}{dx}\Big|_{x=x_0} = y'_c(x_0) + y'_p(x_0) = y'_0
	$$
	Solving these two equations will yield the specific values of $C_1$ and $C_2$ that satisfy the initial conditions, leading to the particular solution for the problem.
	
	
\end{document}
