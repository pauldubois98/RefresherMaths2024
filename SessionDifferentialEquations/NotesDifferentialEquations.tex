\documentclass[]{article}

\usepackage{amsmath}
\usepackage{amsfonts}
\usepackage{amsthm}
\usepackage{amssymb}
\usepackage{mathrsfs}
\usepackage{stmaryrd}

\newcommand{\Q}{\mathbb{Q}}
\newcommand{\N}{\mathbb{N}}
\newcommand{\Z}{\mathbb{Z}}
\newcommand{\R}{\mathbb{R}}
\newcommand{\Primes}{\mathbb{P}}
\newcommand{\st}{\text{ s.t. }}
\newcommand{\txtand}{\text{ and }}
\newcommand{\txtor}{\text{ or }}
\newcommand{\lxor}{\veebar}

%opening
\title{Notes on Differential Equations}
\author{DSBA Mathematics Refresher 2024}
\date{}

\begin{document}
	
	\maketitle
	
	\begin{abstract}
		
	\end{abstract}
	
	\section{Applications of Differential Equations}
	Differential equations play a significant role in various applications within data science and business. These applications demonstrate the versatility of differential equations in modeling complex, dynamic systems across diverse fields, aiding in decision-making and strategic planning.
	\subsection{Predictive Modeling in Finance}
	\paragraph{Black-Scholes Model}\noindent\\
	This is a partial differential equation used to estimate the price of options and other financial derivatives.
	It helps traders and analysts predict the future behavior of asset prices and manage financial risk.
	\subsection{Economics and Market Dynamics}
	\begin{itemize}
		\item \textbf{Macroeconomic Modeling}\noindent\\
		Differential equations are used to model economic growth, inflation, and interest rates.
		For example, the Solow growth model uses differential equations to describe how capital accumulation and technological progress impact economic growth.
		\item \textbf{Epidemic Models}\\
		In business, especially during crises like pandemics, models such as SIR (Susceptible-Infected-Recovered) are used to predict the spread of disease and its impact on markets and workforce productivity.
	\end{itemize}
	\subsection{Customer Behavior and Marketing}
	\begin{itemize}
		\item \textbf{Diffusion Models}\noindent\\
		Differential equations are used to model the adoption of new products and technologies in markets.
		The Bass Diffusion Model, for instance, predicts the rate at which a new product will be adopted over time, helping companies strategize marketing and sales efforts.
		\item \textbf{Churn Prediction}\noindent\\
		Models based on differential equations can predict customer churn by analyzing the rates of customer acquisition and departure over time, allowing businesses to implement targeted retention strategies.
	\end{itemize}
	\subsection{Supply Chain and Inventory Management}
	\paragraph{Dynamic Systems}\noindent\\
	Differential equations are used to model inventory levels over time, helping businesses optimize ordering strategies and reduce holding costs.
	The equations can account for varying demand rates and lead times, improving supply chain efficiency.
	\subsection{Environmental Analysis}
	\paragraph{Time Series Analysis}\noindent\\
	Differential equations are used to model and predict changes in environmental data, such as air quality, temperature, or pollution levels.
	These models help businesses in energy and agriculture make data-driven decisions.
	
	
	
	\section{First-Order Differential Equations}
	
	A first-order differential equation can often be written in the form:
	$$
	\frac{dy}{dx} + P(x) \cdot y = Q(x)
	$$
	To solve this, we use an integrating factor "$\text{IF}$" which is defined as:
	$$
	\text{IF} = e^{\int P(x) \, dx}
	$$
	Multiplying both sides of the differential equation by $\text{IF}$ gives:
	$$
	\text{IF} \cdot \frac{dy}{dx} + \text{IF} \cdot P(x) \cdot y = \text{IF} \cdot Q(x)
	$$
	The left side of the equation is now the derivative of the product $\text{IF} \cdot y$:
	$$
	\frac{d}{dx}\left[\text{IF} \cdot y\right] = \text{IF} \cdot Q(x)
	$$
	Integrating both sides with respect to $x$:
	$$
	\text{IF} \cdot y = \int \text{IF} \cdot Q(x) \, dx + C
	$$
	Finally, solve for $y(x)$:
	$$
	y(x) = \frac{1}{\text{IF}}\left[\int \text{IF} \cdot Q(x) \, dx + C\right]
	$$
	
	\paragraph{Initial Conditions}
	
	Consider the same first-order equation with an initial condition $y(x_0) = y_0$.
	After finding the general solution as in the previous section:
	$$
	y(x) = \frac{1}{\text{IF}}\left[\int \text{IF} \cdot Q(x) \, dx + C\right]
	$$
	Apply the initial condition to find $C$:
	$$
	y(x_0) = \frac{1}{\text{IF}(x_0)}\left[\int \text{IF}Q(x) \, dx \mid_{x_0} + C\right] = y_0
	$$
	Solve for $C$ and conclude.
	
	\section{(Homogeneous) Second-Order Linear Differential Equations}
	
	A second-order homogeneous linear differential equation is of the form:
	$$
	a\frac{d^2y}{dx^2} + b\frac{dy}{dx} + cy = 0
	$$
	The characteristic equation associated with this differential equation is:
	$$
	ar^2 + br + c = 0
	$$
	The nature of the solutions depends on the discriminant $\Delta = b^2 - 4ac$:
	\begin{itemize}
		\item If $\Delta > 0$, the roots $r_1$ and $r_2$ are real and distinct.
		The solution is: $y(x) = C_1e^{r_1x} + C_2e^{r_2x}$.
		\item If $\Delta = 0$, the roots are real and repeated $r_1 = r_2 = r$.
		The solution is: $y(x) = (C_1 + C_2x)e^{rx}$.
		\item If $\Delta < 0$, the roots are complex $r_{1,2} = \alpha \pm i\beta$.
		The solution is: $y(x) = e^{\alpha x}(C_1\cos(\beta x) + C_2\sin(\beta x))$.
	\end{itemize}
	
	\section{Second-Order Linear Differential Equations}
	
	For the general second-order linear differential equation:
	$$
	a\frac{d^2y}{dx^2} + b\frac{dy}{dx} + cy = g(x)
	$$
	The solution is the sum of the complementary function $y_c(x)$
	(the solution to the associated homogeneous equation)
	and a particular solution $y_p(x)$:
	$$
	y(x) = y_c(x) + y_p(x)
	$$
	
	The method for finding $y_p(x)$ depends on the form of $g(x)$:
	\begin{itemize}
		\item If $g(x)$ is a polynomial, use the method of undetermined coefficients.
		\item If $g(x)$ is of an exponential or trigonometric form, use an answer of a similar form.
		\item For other forms of $g(x)$, the method of variation of parameters can be used.
	\end{itemize}
	
	\paragraph{Initial Conditions}
	
	Given the general solution:
	$$
	y(x) = y_c(x) + y_p(x)
	$$
	And initial conditions $y(x_0) = y_0$ and $\frac{dy}{dx}\Big|_{x=x_0} = y'_0$, we determine the constants $C_1$ and $C_2$ in $y_c(x)$:
	$$
	y(x_0) = y_c(x_0) + y_p(x_0) = y_0
	$$
	$$
	\frac{dy}{dx}\Big|_{x=x_0} = y'_c(x_0) + y'_p(x_0) = y'_0
	$$
	Solving these two equations will yield the specific values of $C_1$ and $C_2$ that satisfy the initial conditions, leading to the particular solution for the problem.
	
	
\end{document}
